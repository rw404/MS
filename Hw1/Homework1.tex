\documentclass{article}
\usepackage[utf8]{inputenc}
\usepackage{amsmath}
\usepackage{natbib}
\usepackage{graphicx}
\usepackage[utf8]{inputenc}
\usepackage[english,russian]{babel}
\usepackage{cmap}

\begin{document}

\section{Домашняя работа №1}

\subsection{Задание 1}

С помощью $R[0,1]$ реализовать датчик распределения $Bi(5, \frac{1}{2})$. По смоделированным выборкам длины
100 и 100 000 построить на одном рисунке графики ряда частот и теоретический график. По выборкам
вычислить выборочные среднее и дисперсию, сравнить их с теоретическими значениями.

\subsubsection{Решение}
Для того, чтобы получить распределение $Bi(5, \frac{1}{2})$ из генератора $R[0,1]$, 
возьмем 5 генераторов и полим выборку размерности 5, то есть $X = (X_1, X_2, X_3, X_4, X_5)$, где
$X_i$ - случайная величина, причем $X_i \sim R[0, 1]$. 

Чтобы выполнялось второе условие, то есть "вероятность успеха $\frac{1}{2}$", выберем на отрезке
$[0, 1]$ точку $\frac{1}{2}$, тогда отрезок будет разделен на две равные части. Теперь будем
предпологать, что все значения на полуинтервале $[0, \frac{1}{2})$ 
соответствуют нулю, а $[\frac{1}{2}, 1]$ единице. Вернем сумму полученных генераторов, что
будет той самой случайной величиной $Y \sim Bi(5, \frac{1}{2})$

Рассмотрим графики полученного распределения.

\begin{figure}[h]
\center{
    \includegraphics[scale=0.39]{./Task1/Task_1__all_100.png}
    \includegraphics[scale=0.39]{./Task1/Task_1_Binom100.png}
    \includegraphics[scale=0.39]{./Task1/Task_1__all_100000.png}
    \includegraphics[scale=0.39]{./Task1/Task_1_Binom100000.png}
}
\caption{Распределение Бернулли}
\label{fig:image}
\end{figure}

Вычислим выборочное среднее:

    \[X = (X_1, X_2, ..., X_n); n = \{100, 100000\}; 
    E_X = \sum_{i = 0}^{n}\frac{1}{n} * X_i\]

И дисперсию:
    \[X = (X_1, ..., X_n); n = \{100, 100000\};
    D_X = E_(X^2) - (E_X)^2\]

Для выборки длины 100:
\begin{itemize}
    \item Выборочное среднее: $2.6$, теоретическое: $2.52$
    \item Дисперсия смоделированной выборки: $1.38$, теоретическое: $1.1095999999999995$
\end{itemize}

Для выборки длины 100000:
\begin{itemize}
    \item Выборочное среднее: $2.50395$, теоретическое: $2.498$
    \item Дисперсия смоделированной выборки: $1.2560643974999994$, теоретическое: $1.2472359999999991$
\end{itemize}

\subsection{Задание 2}
С помощью $R[0,1]$ реализовать датчик стандартного распределения Коши. По смоделированным выборкам длины 100 и 100 000 построить на одном рисунке теоретических и эмпирических функций распределения. На одном рисунке построить гистограммы смоделированных выборок (число и положение разрядов выбрать самостоятельно, но осмысленно!) и теоретическую плотность распределения. Вычислить выборочные медианы и сравнить их с теоретическим значением.

\subsubsection{Решение}
Для моделлирования случайных величин некоторого распределения через генератор 
равномерного распределение рассмотрим функцию, обратную функции распределения Коши: 
\[ F^{-1}(X) = \tg\left({\pi*(X-\frac{1}{2})}\right), X \sim R[0,1]\]
По решению задачи №7 полученная случайная величина $Y = F^{-1}(X)$ распределена 
по закону Коши, то есть $Y \sim C(0, 1)$. Для вычисления эмперической функции
распределения используем формулу:
\[F_n(x) = \frac{1}{n}*\sum_{k=1}^{n}I(x-X_k)\]
\begin{equation*}
    I(x)=
    \begin{cases}
        0 &\text{при $x \le 0$,}\\
        1 &\text{при $x < 0$.}
    \end{cases}
\end{equation*}

Для построения теоретического распределения воспользуемся распределением $C(0, 1)$:
\[F(x) = \frac{1}{\pi}*\arctg\left(x\right)+\frac{1}{2}\]

Рассмотрим полученные графики для выборки длины 100 и 100000:

\begin{figure}[h]
\center{
    \includegraphics[scale=0.39]{./Task2/Task_2__all_100.png}
    \includegraphics[scale=0.39]{./Task2/Task_2_Cauchy100.png}
    \includegraphics[scale=0.39]{./Task2/Task_2__all_100000.png}
    \includegraphics[scale=0.39]{./Task2/Task_2_Cauchy100000.png}
}
\caption{Распределение Коши}
\label{fig:image}
\end{figure}

Построим гистограмму и плотность распределения Коши. Для плотности справедлива формула:
\[f(x) = \frac{1}{\pi*\left(1+x^2\right)}\]

Чтобы построить гистограмму,  определим число разрядов и выберем отрезок так, чтобы
выполнялось условие нормировки. Данный вопрос рассмотрен в задаче №3, поэтому выберем отрезок
так, что левая границы соответствует минимуму из полученной выборки, а правая - максимуму:
\[X = \left(X_1, ..., X_n\right), [a,b] = \left[\min_{i \in \overline{1,n}}X_i,
\max_{i \in \overline{1, n}}{X_i}\right]\]

Для определения числа разрядов гистограммы воспользуемся правилом Стёрджеса. Число разрядов
$n$ определяется через размер выборки $N$:
\[n(N) = 1+\left\lfloor{\log_2N}\right\rfloor\]

Используя полученные значения, построим плотность и гистограмму распределения Коши:

\begin{figure}[h]
\center{
    \includegraphics[scale=0.5]{./Task2/Task_2_Density100.png} 
    \includegraphics[scale=0.5]{./Task2/Task_2_Density100000.png}
}
\caption{Плотность и гистограмма распределения Коши}
\label{fig:image}
\end{figure}

Вычислим выборочную медиану и сравним значение с теоретическим аналогом. Теоретическое значение
получается при подстановке в обратную функцию распределения число $\frac{1}{2}$:

\[M = F^{-1}(\frac{1}{2})\]

\begin{itemize}
    \item Для выборки длины 100 выборочная медиана: 0.21787135837578478; Теоретическое значение: 0
    \item Для выборки длины 100000 выборочная медиана: -0.0005758960784728971; Теоретическое значение: 0
\end{itemize}

\subsection{Задание 3}
В условиях задачи 8 выбрать произвольную кусочно-постоянную плотность с 5 разрядами. 
помощью $R[0,1]$ реализовать соответствующий генератор. По смоделированным выборкам длины
1000 и 100 000 построить на одном рисунке теоретических и эмпирических функций
распределения. По выборкам вычислить выборочные среднее и дисперсию, сравнить их с 
теоретическими значениями. На одном рисунке построить гистограммы смоделированных выборок
и теоретическую плотность распределения. Разряды гистограмм выбирать двумя способами: 1)
совпадающими с теоретическими разрядами распределения, 2) разбивающими теоретические 
разряды пополам. По выборкам вычислить выборочные среднее и дисперсию, сравнить их с
теоретическими значениями.
\subsubsection{Решение}
Чтобы смоделировать кусочно-постоянную плотность распределения, рассмотрим отрезок 
$[0, 1]$. На данном отрезке возьмем 5 точек, не равных нулю, и ноль. Например:
\[[0, 1] \to \left[0, \frac{5}{16}\right) \cup \left[\frac{5}{16}, \frac{7}{16}\right)
\cup \left[\frac{7}{16}, \frac{10}{16}\right) \cup \left[\frac{10}{16},
\frac{11}{16}\right) \cup \left[\frac{11}{16}, 1\right) \]

Теперь мы получили разбиение отрезка на 5 частей. Выберем на каждом плотность так, чтобы
выполнялось условие нормировки. Положим, что:

\begin{equsation*}
    f(x) =     
    \begin{cases}
        0 &\text{при $x < 0$,}\\
        1 &\text{при $0 \le x < \frac{5}{16}$,}\\
        \frac{1}{2} &\text{при $\frac{5}{16} \le x < \frac{7}{16}$,}\\
        \frac{2}{3} &\text{при $\frac{7}{16} \le x < \frac{10}{16}$,}\\
        2 &\text{при $\frac{10}{16} \le x < \frac{11}{16}$,}\\
        \frac{6}{5} &\text{при $\frac{11}{16} \le x < 1$,}\\
        0 &\text{при $1 \le x$.}
    \end{cases}
\end{equation*}

Проверим условие нормировки:
\[\int_{-\infty}^{+\infty}f(x)dx = \frac{5}{16}\times1+\frac{2}{16}\times\frac{1}{2}+
\frac{3}{16}\times\frac{2}{3}+\frac{1}{16}\times2+\frac{5}{16}\times\frac{6}{5}
= 1\]

Поэтому выбранное разбиение верное и у полученной случайной величины кусочно-постоянная плотность. 
Чтобы смоделировать генератор полученной случайной величины используем два генератора
\begin{itemize}
    \item Первый генератор выдаст точку на отрезке $[0, 1]$. Используем наше разбиение отрезка на
        5 частей и определим, какому множеству принадлежит полученная точка
    \item Второй генератор покажет на какую позицию разбиения попадет точка. Уже
        определено, какому отрезку принадлежит данная точка, известна длина отрезка и значение
        левой границы. Найденные значения подставим в формулу случайной величины:
        \[X() = \sum_{i = 1}^5|X_i - X_{i-1}|*I_{\Delta\left[X_{i-1}, X_{i}\right]},
        X_0 = 0\]
\end{itemize}

Например, если после работы первого генератора точка находится на множестве $\left[0,
\frac{5}{16}\right)$, то позиция точки принадлежит этому же интервалу, тогда второй генератор
дает число от 0 до 1, которое умножается на длину замыкания данного множества, то есть на
$\frac{5}{16}$, затем прибавляется начало множества - точка 0. 

По определению распределения получим, что
\[F_X(x) = \int_{-\infty}^xf(x)dx\]

Так как плотность кусочно-постоянная, то функция распределения будет линейной функцией с
множителями перед $x$, соответствующими значению плотности на данном множестве, сложенная с
константой, равной $\int_{-\infty}^{a_l}f(x)dx$, где $a_l$ - левая граница данного множества.

Так как в точке $\frac{7}{16}$ плотность меняется с $\frac{1}{2}$ на $\frac{2}{3}$, для большей
наглядности добавленая вспомогательная функция $y(x) = \frac{2}{3}*x+\frac{1}{12}$,
показывающая, что перелом в точке $\frac{7}{16}$, хоть и незначительный, есть.

Для эмперической функции распределения используем уже упомянутую в предыдущих заданиях этого
документа формулу. Получим:

\begin{figure}[h]
\center{
    \includegraphics[scale=0.39]{./Task3/Task_3__all_100.png}
    \includegraphics[scale=0.39]{./Task3/Task_3__distrib_100.png}
    \includegraphics[scale=0.39]{./Task3/Task_3__all_100000.png}
    \includegraphics[scale=0.39]{./Task3/Task_3__distrib_100000.png}
}
\caption{Распределение с кусочно-постоянной плотностью}
\label{fig:image}
\end{figure}

Сравним полученные выборочные среднее и дисперсию, сравним их с еоретическими значениями:
\begin{itemize}
    \item Для выборки длины 100 выборочное среднее составило: 0.4252; теоретическое - 0.4632670454545454
    \item Для выборки длины 100000 выборочное среднее составило: 0.4633432522;
        теоретическое - 0.462890996097461
    \item Для выборки длины 100 выборочная дисперсия составила: 0.07501495999999996;
        теоретическая - 0.07843229629324083
    \item Для выборки длины 100000 выборочная дисперсия составила: 0.0766235435949392;
        теоретическая - 0.07666826120296805
\end{itemize}

При построении гистограммы использовалось два способа, как описано в условии задачи. 
Плотность данной случайной величины, как следует из требования задачи, кусочно-постоянная функция с
значениями указанными ранее на множествах описанных выше. Число разрядов фиксированно и в 1-ом
случае равно 5(по условию задачи), во втором случае равно 10.

Для построения гистограммы использовалась описанная ранее в заданиях этого документа формула.
Для выбора отрезка, на котором строится гистограмма, использовалась ранее описанная идея с
максимумом и минимумом выборки. Получим:

\begin{figure}[h]
\center{
    \includegraphics[scale=0.39]{./Task3/Task_3__density1_100.png}
    \includegraphics[scale=0.39]{./Task3/Task_3__density2_100.png}
    \includegraphics[scale=0.39]{./Task3/Task_3__density1_100000.png}
    \includegraphics[scale=0.39]{./Task3/Task_3__density2_100000.png}
}
\caption{Плотность и гистограмма распределения с кусочно-постоянной плотностью}
\label{fig:image}
\end{figure}

\subsection{Задание 4}


\end{document}

