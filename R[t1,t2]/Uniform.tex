\documentclass{article}
\usepackage[utf8]{inputenc}
\usepackage{amsmath}
\usepackage{natbib}
\usepackage{graphicx}
\usepackage[utf8]{inputenc}
\usepackage[english,russian]{babel}
\usepackage{cmap}
\usepackage{float}
\usepackage{amsmath,amssymb}
\usepackage[russian]{babel}

\begin{document}
	
\section{Задача 1}
Построить аисптотический доверительный интервал параметров равномерного распределения $\mathbb{R}[\theta_1, \theta_2]$
\subsection{Решение}

Пусть дана выборка $\mathbb{X} = \left(X_1, ..., X_n\right) \sim \mathbb{R}[\theta_1; \theta_2]$
\begin{itemize}
	\item Считаю 1-ый и 2-ой начальные моменты равномерного распределения:
\end{itemize}

\[\mathbb{E}[X_i] = \frac{\theta_1 + \theta_2}{2}\]
\[\mathbb{E}[X_i^2] = \mathbb{D}[X_i]+\left(\mathbb{E}X_i\right)^2=\frac{(\theta_2-\theta_2)^2}{12}+
\frac{(\theta_2+\theta_1)^2}{4}=\frac{\theta_2^2-2\theta_1\theta_2+\theta_1^2}{12}+\]
\[+\frac{3(\theta_2^2+2\theta_1\theta_2+\theta_1^2)}{12}=\]
\[=\frac{4\theta_2^2+4\theta_1\theta_2+4\theta_1^2}{12}=\frac{\theta_2^2+\theta_1\theta_2+\theta_1^2}{3}\]
\begin{equation*}
	\begin{cases}
		\overline{X} = \mathbb{E}[X]\\
		S_{n,2} = \mathbb{E}[X^2]
	\end{cases}
\end{equation*}

\begin{itemize}
	\item Доказываю их асимптотическую нормальность и нахожу математическое ожидание и ковариационную матрицу:
\end{itemize}

Для доказательства воспользуюсь центральной предельной теоремой:
\[\overline{X}\to \theta = \frac{\theta_1+\theta_2}{2}<\infty\]
\[\mathbb{D}\left[\overline{X}\right] = \frac{\left(\theta_2-\theta_1\right)^2}{12}<\infty\]
Тогда \[\sqrt{n}\frac{\overline{X}-\theta}{\sqrt{\mathbb{D}\left[\overline{X}\right]}}\to\mathbb{N}[0, 1]\]

Теперь рассмотрим выборочный начальный момент второго порядка $S_{n,2}$:
\[\mathbb{E}[S_{n,2}] = \mathbb{E}\left[\sum_{i = 1}^{n}\frac{1}{n}X_i^2\right]=\frac{1}{n}\sum_{i=1}^{n}\mathbb{E}[X_i^2]=\frac{1}{n}n\mathbb{E}[X_i^2]=\]
\[=\mathbb{E}[X_i^2];\]

То есть $S_{n,2}\to\mathbb{E}[X_i^2]$, тогда по теореме непрерывности 
$\mathbb{D}(S_{n,2})\xrightarrow{\text{п.н.}}\mathbb{D}\left[\mathbb{E}(X_i^2)\right]$ (в том же смысле, что и
$S_{n,2}\xrightarrow{\text{п.н.}}\mathbb{E}[X_i^2]$). 
Получаем конечность двух первых начальных моментов, тогда применима ЦПТ и асимптотическая нормальность доказана.

Теперь подсчитаем математическое ожидание и ковариационную матрицу вектора выборочных моментов:

\[\begin{pmatrix} \overline{X} \\ S_{n,2} \end{pmatrix}\]

Вектор матемтических ожиданий получается из несмещенности рассмотренных выше статистик:
\[m = \begin{pmatrix}
	\frac{\theta_2+\theta_1}{2}\\
	\frac{\theta_2^2+\theta_1\theta_2+\theta_1^2}{3}
\end{pmatrix}\]

Для подсчета ковариационной матрицы воспользуемся тем фактом, что начальный момент n-го порядка Равномерного распределения на отрезке [$\theta_1$, $\theta_2$] считается так:
\[\mathbb{E}[X_i^n] = \frac{\theta_2^{n+1} - \theta_1^{n+1}}{(n+1)(\theta_2-\theta_1)}\]
и что элементы выборки независимы.

Тогда 
\[\textbf{$\mathbb{D}[\overline{X}]$} = \frac{1}{n}\mathbb{D}[X_i] = \frac{(\theta_2-\theta_1)^2}{12n}\]
\[\textbf{$\mathbb{D}[S_{n,2}]$} = \mathbb{D}\left[\frac{1}{n}\sum_{i=1}^{n}X_i^2\right] = \frac{1}{n}\mathbb{D}[X_i^2] = \frac{1}{n}\left(\mathbb{E}[X_i^4]-\left(\mathbb{E}[X_i^2]\right)^2\right)=\]
\[=\frac{1}{n}\left(\frac{\theta_2^5-\theta_1^5}{5(\theta_2-\theta_1)}-\left(\frac{\theta_2^{3} - \theta_1^{3}}{3(\theta_2-\theta_1)}\right)^2\right)\]
\[\textbf{$cov(\overline{X}, S_{n,2})$} = \mathbb{E}\left[\stackrel{o}{\overline{X}}\stackrel{o}{S_{n,2}}\right]=\mathbb{E}\left[\frac{1}{n}\sum_{i=1}^{n}\stackrel{o}{X_i}\frac{1}{n}\sum_{j=1}^{n}\stackrel{o}{X_j^2}\right]=\]
\[=\frac{1}{n^2}\sum_{i=1,j=1}^{n}\mathbb{E}\left[\stackrel{o}{X_i}\stackrel{o}{X_j^2}\right]=\frac{n}{n^2}\mathbb{E}\left[\stackrel{o}{X_i^3}\right]=\left|\mu = \mathbb{E}[X_i]\right|=\]
\[=\frac{1}{n}\mathbb{E}\left[(X_i-\mu)^3\right]=\frac{1}{n}\left(\mathbb{E}[X_i^3]-3\mu\mathbb{E}[X_i^2]+3\mu^2\mathbb{E}[X_i]-\mu^3\right)=\]
\[=\frac{1}{n}\left(\frac{\theta_2^{4} - \theta_1^{4}}{4(\theta_2-\theta_1)}-3\mu\frac{\theta_2^{3} - \theta_1^{3}}{3(\theta_2-\theta_1)}+3\mu^3-\mu^3\right)\]


Введем обозначения для элементов матрицы:
\begin{equation*}
	\begin{cases}
		\sigma^2_{\overline{X}} &= \textbf{$\mathbb{D}[\overline{X}]$}\\
		c & = cov\left(\overline{X}, S_{n, 2}\right) \\
		\sigma^2_{S_{n, 2}} &= \textbf{$\mathbb{D}[S_{n, 2}]$}\\
	\end{cases}
\end{equation*}

Тогда используя полученные выше выражения справедливо, что 

\[
K(\theta_1, \theta_2) = 
\begin{pmatrix}
	\sigma^2_{\overline{X}} & c\\
	c & \sigma^2_{S_{n, 2}}
\end{pmatrix}\]

Теперь получим представления для элементов этой матрицы через выбранные статистики.

\[\textbf{$\mathbb{D}\left[\overline{X}\right]$} = \frac{1}{n}\left(\textbf{$\mathbb{D}[X]$}\right) = \frac{1}{n}(\mathbb{E}\left[X_i^2\right]-\left(\mathbb{E}[X_i]\right)^2) = \frac{1}{n}\left(S_{n, 2} - \left(\overline{X}\right)^2\right);\]

Выразим $\mathbb{E}\left[X_i^4\right]$ через статистики:
\[\mathbb{E}\left[X_i^4\right] = \frac{\theta_2^4+\theta_2^3\theta_1+\theta_2^2\theta_1^2+\theta_2\theta_1^3+\theta_1^4}{5};\]
\[9(S_{n, 2})^2 = (\theta_2^2+\theta_2\theta_1+\theta_1^2)^2 = \theta_2^4+\theta_2^2\theta_1^2+\theta_1^4 + 2\theta_2^3\theta_1+2\theta_2^2\theta_1^2+2\theta_2\theta_1^3 = 5\mathbb{E}\left[X_i^4\right] + \]
\[+\theta_1\theta_2(\theta_2^2+2\theta_2\theta_1+\theta_1^2) = 5\mathbb{E}\left[X_i^4\right] + 4\theta_2\theta_1\left(\overline{X}^2\right);\]
\[\theta_2\theta_1 = 4\overline{X}^2-3S_{n, 2};\]
\[5\mathbb{E}\left[X_i^4\right] = 9(S_{n, 2})^2 - 4(4\overline{X}^2-3S_{n, 2})\overline{X}^2;\]
Тогда 
\[\mathbb{D}[S_{n, 2}] = \frac{1}{n}\left(\mathbb{E}\left[X_i^4\right]- \left(\mathbb{E}[X_i^2]\right)^2\right) = \frac{1}{n}\left(\frac{9(S_{n, 2})^2 - 4(4\overline{X}^2-3S_{n, 2})\overline{X}^2}{5} - S_{n, 2}^2\right) = \]
\[ = \frac{1}{5n}\left(4(S_{n, 2})^2 - 4(4\overline{X}^2-3S_{n, 2})\overline{X}^2\right);\]

Наконец, для нахождения $cov\left(\overline{X}, S_{n, 2}\right)$ запишем выражение $\mathbb{E}\left[X_i^3\right]$ через статистики:
\[\mathbb{E}\left[X_i^3\right] = \frac{\theta_2^4-\theta_1^4}{4(\theta_2-\theta_1)}=\frac{(\theta_2^2-\theta_1^2)(\theta_2^2+\theta_1^2)}{4(\theta_2-\theta_1)} = \frac{(\theta_2+\theta_1)(\theta_2^2+\theta_1^2)}{4} = \frac{\theta_2^2+\theta_1^2}{2}\frac{\theta_2+\theta_1}{2}=\]
\[=\frac{\theta_2^2+\theta_1^2}{2}\overline{X};\]
\[\theta_2^2+\theta_1^2 = 6S_{n,2}-4\overline{X}^2;\]
\[\mathbb{E}\left[X_i^3\right] = \left(3S_{n, 2}-2\overline{X}^2\right)\overline{X};\]

\[cov\left(\overline{X}, S_{n, 2}\right)=\frac{1}{n}\left(\frac{\theta_2^{4} - \theta_1^{4}}{4(\theta_2-\theta_1)}-3\mu\frac{\theta_2^{3} - \theta_1^{3}}{3(\theta_2-\theta_1)}+3\mu^3-\mu^3\right) = \left|\overline{X} = \mu\right| = \]
\[ = \frac{1}{n}\left(3S_{n, 2}\overline{X} - 2\overline{X}^3-3\overline{X}S_{n,2}+3\overline{X}^3-\overline{X}^3\right) = 0\]

Таким образом, получена ковариационная матрица, зависящая от статистик:

\[K(T) = K\left(\overline{X}, S_{n, 2}\right) = \frac{1}{n}
\begin{pmatrix}
	S_{n, 2} - \left(\overline{X}\right)^2 & 0\\
	0 & \frac{1}{5}\left(4(S_{n, 2})^2 - 4(4\overline{X}^2-3S_{n, 2})\overline{X}^2\right)
\end{pmatrix}\]
\begin{itemize}
	\item На занятиях были посчиталны функции, которые выражают концы отрезка через моменты.
\end{itemize}

Полученные на семинаре нелинейные функции $h_i(x), i = \{0, 1\}$ имеют следующий вид:
\begin{equation*}
	\begin{cases}
		h_1\left(\overline{X}, S_{n, 2}\right) = \overline{X}-\sqrt{3\left(S_{n, 2}-(\overline{X})^2\right)}\\
		h_2\left(\overline{X}, S_{n, 2}\right) = \overline{X}+\sqrt{3\left(S_{n, 2}-(\overline{X})^2\right)}
	\end{cases}
\end{equation*}

Далее будем предполагать, что 
\[M = 
\begin{pmatrix}
	\mathbb{E}[X] \\
	\mathbb{E}\left[X^2\right] 
\end{pmatrix}\]

\[T = 
\begin{pmatrix}
	\overline{X}\\
	S_{n, 2}
\end{pmatrix}\]
В данном случае результат нелинейного преобразования асимптотически гауссовского вектора будет асимптотически гауссовским.

Так как $\mathbb{L}_\theta\left(\sqrt{n}\left(T-M\right)\right) \to \mathbb{N}\left(0, K(\theta)\right)$, тогда, положив, что
\[\phi(T) = 
\begin{pmatrix}
	h_1(T) \\
	h_2(T)
\end{pmatrix}\] получаем $\mathbb{L}_\theta\left(\sqrt{n}\left(\phi(T)-\phi(M)\right)\right) \to \mathbb{N}\left(0, v^2(\theta)\right)$, где 

\[v^2(\theta) = b^T(\theta)K(\theta)b(\theta), b(\theta) = 
\begin{pmatrix}
	\frac{\partial{\phi(\theta)}}{\partial{\theta_1}}\\
	\frac{\partial{\phi(\theta)}}{\partial{\theta_2}}
\end{pmatrix}
=
\begin{pmatrix}
	\frac{\partial{h_1(\theta)}}{\partial{\theta_1}} & \frac{\partial{h_2(\theta)}}{\partial{\theta_1}}\\
	\frac{\partial{h_1(\theta)}}{\partial{\theta_2}} & \frac{\partial{h_2(\theta)}}{\partial{\theta_2}}
\end{pmatrix}
=
\begin{pmatrix}
	\frac{1}{2}+\frac{1}{2\sqrt{3}} & \frac{1}{2}-\frac{1}{2\sqrt{3}}\\
	\\
	\frac{1}{2}-\frac{1}{2\sqrt{3}} & \frac{1}{2}+\frac{1}{2\sqrt{3}}
\end{pmatrix}\]

Также заметим, что 
\[\theta = 
\begin{pmatrix}
	\theta_1\\
	\theta_2
\end{pmatrix} = \phi(M)\]
Отсюда и из установленного ранее тождества $c = cov(\overline{X}, S_{n, 2}) = 0$ получаем, что 
\[b^T(\theta)K(\theta) = 
\begin{pmatrix}
	\frac{\sqrt{3}+1}{2\sqrt{3}}\sigma^2_{\overline{X}} & \frac{\sqrt{3}-1}{2\sqrt{3}}\sigma^2_{S_{n, 2}}\\
	\\
	\frac{\sqrt{3}-1}{2\sqrt{3}}\sigma^2_{\overline{X}} & \frac{\sqrt{3}+1}{2\sqrt{3}}\sigma^2_{S_{n, 2}}
\end{pmatrix}\]

\[v^2(\theta) = b^T(\theta)K(\theta)b(\theta), b(\theta) = 
\begin{pmatrix}
	\frac{(\sqrt{3}+1)^2}{4*3}\sigma^2_{\overline{X}}+\frac{(\sqrt{3}-1)^2}{4*3}\sigma^2_{S_{n,2}} & \frac{2}{12}\sigma^2_{\overline{X}}+\frac{2}{12}\sigma^2_{S_{n,2}}\\
	\frac{2}{12}\sigma^2_{\overline{X}}+\frac{2}{12}\sigma^2_{S_{n,2}} & \frac{(\sqrt{3}+1)^2}{4*3}\sigma^2_{\overline{X}}+\frac{(\sqrt{3}-1)^2}{4*3}\sigma^2_{S_{n,2}}
\end{pmatrix}\]

Тогда $\mathbb{L}_\theta\left(\sqrt{n}\frac{\left(\phi(T)-\phi(M)\right)}{v(T)}\right) \to \mathbb{N}(0, I)$

Используя равенство $\theta = \phi(M)$, получаем ответ:
\[\theta = 
\begin{pmatrix}
	\theta_1\\
	\theta_2
\end{pmatrix}
\in\left(h(T)-\frac{v(T)c_\gamma}{\sqrt{n}}; h(T) + \frac{v(T)c_\gamma}{\sqrt{n}}\right), \]
\[\text{где }c_\gamma = 
\begin{pmatrix}
	(c_\gamma)_1\\
	(c_\gamma)_2
\end{pmatrix} \\- \text{многомерный квантиль} \gamma \text{ стандартного нормального распределения}\]

\end{document}